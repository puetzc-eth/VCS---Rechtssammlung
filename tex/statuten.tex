\section{Name, Zweck}

\subsection{Name}
\begin{enumerate}
\item Unter dem Namen "Vereinigung der Studierenden der Chemie, Biochemie – Chemische Biologie, Chemieingenieurwissenschaften und Interdisziplinären Naturwissenschaften an der ETH Zürich”, auch "Vereinigung der Chemiestudierenden" genannt, abgekürzt „VCS“, besteht als Verein im Sinne von Artikel 52ff und 60ff des ZGB eine autonome Sektion des Verbands der Studierenden an der ETH (VSETH) gemäss Artikel 14ff der Statuten des VSETH.
 \item Die Statuten des VSETH sind denjenigen der VCS übergeordnet.
\end{enumerate}

\subsection{Zweck, Tätigkeit}
\begin{enumerate}
\item Der Verein bezweckt:
	\begin{itemize}
	\item die Wahrung der materiellen und geistigen Interessen der Mitglieder
	\item die Teilnahme an der wissenschaftlichen und hochschulpolitischen Auseinandersetzung
	\item die Förderung der Kommunikation und des Austausches innerhalb des Departements für Chemie und angewandte Biowissenschaften (D-CHAB), der ETH und zu anderen Hochschulen
	\item die Förderung des Kontakts mit der branchenspezifischen Industrie
	\end{itemize}
\item Dieser Zweck wird insbesondere verfolgt durch:
	\begin{itemize}
	\item die Mitarbeit in den Gremien und Organen des D-CHAB und des VSETH
	\item die Herausgabe eines Informationsorgans
	\item die Durchführung von Exkursionen und Anlässen
	\item die Beschaffung von Prüfungen des D-CHAB
	\item den Kontakt mit anderen Fachvereinen
	\end{itemize}
\item Der Verein untersagt sich politische oder religiöse Aktivitäten, die nicht im Zusammenhang mit seinen Interessen stehen.
\item Der Sitz der VCS ist die ETH Hönggerberg, Zürich.
\item Die VCS achtet bei Durchführung sämtlicher Aktivitäten darauf, diese nachhaltig und umweltschonend durchzuführen.
\end{enumerate}

\subsection{Geschäftsjahr}
\begin{enumerate}
\item Die Geschäftsperiode dauert vom 1. Januar bis zum 31. Dezember. Sie umfasst eine Abrechnungsperiode. Sie unterteilt sich in die folgenden Quartale:
	\begin{itemize}
	\item 1. Quartal: 1. Januar bis 31. März
	\item 2. Quartal: 1. April bis 30. Juni
	\item 3. Quartal: 1. Juli bis 30. September
	\item 4. Quartal: 1. Oktober bis 31. Dezember
	\end{itemize}
\end{enumerate}

\section{Mitglieder}

\subsection{Mitgliedschaft}
%2. 3. Ehrenmitglieder sind nicht-ordentliche Mitglieder im Sinne von Art. 13 Abs. 4bis
%der VSETH-Statuten. Natürliche Personen, denen die ordentliche Mitgliedschaft in
%der VCS nicht offen steht, können die Ehrenmitgliedschaft durch Nomination durch
%ein stimmberechtigtes VCS-Mitglied erlangen. Über die Aufnahme entscheidet die
%Generalversammlung.
%4. Ehrenmitglieder werden durch den Beitritt zur VCS nicht Mitglied im VSETH.
%5. Passivmitglieder sind nicht-ordentliche Mitglieder im Sinne von Art. 13 Abs. 4bis
%der VSETH-Statuten, welch
\begin{enumerate}
\item Der Verein kennt ordentliche Mitglieder, sowie Ehren- und Passivmitglieder.
\item Ordentliche Mitglieder sind alle VSETH-Mitglieder derjenigen Studiengänge, welche gemäss Art. 1ff des Reglements über die Fachvereine des VSETH in der Zuordnungsliste der VCS zugeteilt sind.
\item Ehrenmitglieder sind ausserordentliche Mitglieder im Sinne von Art. 3 des Reglements über die Fachvereine des VSETH. Natürliche Personen, denen die ordentliche Mitgliedschaft in der VCS nicht offen steht, können die Ehrenmitgliedschaft durch Nomination durch ein stimmberechtigtes VCS-Mitglied erlangen. Ordentliche Mitglieder können die Ehrenmitgliedschaft ebenfalls durch Nomination durch ein stimmberechtigtes VCS-Mitglied erlangen. Falls angenommen gilt die Ehrenmitgliedschaft ab dem Zeitpunkt wo die gewählte Person nicht mehr ordentliches Mitglied in der VCS ist. Über die Aufnahme entscheidet die Generalversammlung. Bei mehreren zu behandelnden Äntragen auf Ehrenmitgliedschaft werden diese \textit{in corpore} gewählt, es sei denn, die Generalversammlung beschliesst einen anderen Wahlmodus.
\item Ehrenmitglieder werden durch den Beitritt zur VCS nicht Mitglied im VSETH.
\item Passivmitglieder sind ausserordentliche Mitglieder im Sinne von Art. 3 des Reglements über die Fachvereine des VSETH, welche bereits Mitglied im VSETH sind. Über die Aufnahme entscheidet der Vorstand.
\item Der Vorstand präsentiert eine Liste der seit der vorhergehenden Generalversammlung neu aufgenommenen Passivmitglieder an der Generalversammlung.
\end{enumerate}

\subsection{Rechte}
\begin{enumerate}
\item Alle ordentlichen Mitglieder besitzen das aktive und das passive Wahlrecht, Stimmrecht, sowie das Recht, dem Vorstand, der Generalversammlung sowie allen anderen Organen Anträge zu machen.
\item Lediglich ordentliche Mitglieder, die Bachelor- und Masterstudierende, sowie Studierende der didaktischen Ausbildung der ETH Zürich sind, besitzen passives Wahlrecht für Vertretungen gemäss Abschnitt \ref{sec:Vertretungen} und die Vorstandsämter Präsidium, Quästur und Hochschulpolitik. Sämtliche ordentlichen und ausserordentlichen Mitglieder besitzen passives Wahlrecht für andere Vorstandsämter, wobei nicht mehr als zwei ausserordentliche Mitglieder zeitgleich im Vorstand tätig sein können. Sie haben an allen Versammlungen und Vereinsanlässen freies Wort.
\item Die Mitglieder haben zu jeder Zeit das Recht, alle Vereinsakten, insbesondere Protokolle von Generalversammlungen und Vorstandssitzungen, einzusehen.
\item Alle Mitglieder geniessen sämtliche Vorteile des Vereins. Es liegt im Ermessen des Vorstandes, inwiefern seine Aktivitäten auch Nichtmitgliedern offen stehen sollen.
\end{enumerate}


\subsection{Pflichten}
\begin{enumerate}
\item Jedes Mitglied ist verpflichtet, dem Vereinszweck nicht entgegen zu wirken.
\item Jedes Mitglied ist gehalten, an den Fachvereinsversammlungen, den Wahlen und Abstimmungen
teilzunehmen.
\item Jedes Mitglied ist verpflichtet, die von ihm übernommenen Arbeiten genau auszuführen.
\end{enumerate}

\subsection{Austritt und Ausschluss}
\begin{enumerate}
\item Mit dem Austritt oder Ausschluss aus dem VSETH erlischt automatisch die ordentliche Mitgliedschaft und die Passivmitgliedschaft in der VCS.
\item Bei Nichtbezahlen des Semesterbeitrages wird automatisch der Austritt vermutet.
\item Auf begründeten Antrag kann ein Passiv- oder Ehrenmitglied durch die Generalversammlung aus der VCS ausgeschlossen werden. Als Grund gelten unter anderem
	\begin{itemize}
	\item vorsätzliche Verletzungen der Statuten oder Beschlüsse des Vereins
	\item vorsätzliche Schädigung der Interessen der VCS oder des VSETH.
	\end{itemize}
\item  Jedes Mitglied der VCS ist berechtigt einen Antrag auf Ausschluss zu stellen.
\item Das Mitglied ist 10 Tage vor der Generalversammlung über den Ausschlussantrag durch einen eingeschriebenen Brief zu informieren.
\end{enumerate}

\subsection{Mitgliederbeitrag}
\begin{enumerate}
\item Die Mitgliederbeiträge der ordentlichen Mitglieder und somit VSETH-Mitglieder werden ausschliesslich vom VSETH festgelegt und erhoben.
\item Der Mitgliederbeitrag der Passivmitglieder beträgt 10 CHF pro Semester und wird direkt an die VCS bezahlt.
\item Auf Ehrenmitglieder wird kein Mitgliederbeitrag erhoben.
\end{enumerate}


\section{Finanzen}

\subsection{Mittel}
\begin{enumerate}
\item Die Einnahmen der VCS bestehen grundsätzlich aus den ihr vom VSETH zugewiesenen Mitteln, den Mitgliederbeiträgen der Passivmitglieder und den Zinsen auf das Vereinsvermögen.
\item Die VCS kann sich weitere Einnahmequellen erschliessen. Als solche gelten unter anderem Gönnerbeiträge, Überschüsse aus Dienstleistungen und durchgeführten Veranstaltungen sowie Sponsorenbeiträge.
\end{enumerate}

\subsection{Fonds}
\begin{enumerate}
  \item Fonds haben die Aufgabe, finanzielle Mittel für einen spezifischen Zweck bereitzustellen.
  \item Fonds werden durch die Generalversammlung eröffnet und aufgelöst.
  \item Mit der Eröffnung eines Fonds beschliesst die Generalversammlung ein Fondsreglement. Dieses regelt mindestens folgende Punkte:
    \begin{itemize}
      \item Verwendungszweck
      \item Zuständiges Gremium
      \item Auflösung
    \end{itemize}
  \item Alle existierenden Fondsreglemente sind Bestandteil des Reglements des zuständigen Gremiums und sind im Appendix des Reglements aufzuführen.
\end{enumerate}

\subsection{Haftung}
\begin{enumerate}
\item Für Verbindlichkeiten der VCS haftet nur das Vereinsvermögen. Die Haftung der Mitglieder ist auf den Mitgliederbeitrag beschränkt.
\end{enumerate}

\section{Organe}

\subsection{Organe}
\begin{enumerate}
\item Die Organe des Vereins sind:
	\begin{itemize}
	\item die Generalversammlung
	\item der Vorstand
	\item die Rechnungsrevisoren
	\item die Kommissionen
	\item die Vertretungen in anderen Organisationen
	\item die Fachvereinszeitschrift (\textit{Exsikkator})
	\end{itemize}
\item Die Generalversammlung ist das höchste Organ der VCS.
\end{enumerate}

\subsection{Ordentliche Generalversammlung}
\begin{enumerate}
\item Jedes Semester muss eine ordentliche Generalversammlung vom Vorstand einberufen werden. Sie muss spätestens 7 Tage vor Semesterende stattfinden.
\item Ordentliche Generalversammlungen müssen mindestens 7 Tage im Voraus durch geeignete Bekanntmachung angekündigt werden.
\end{enumerate}

\subsection{Ausserordentliche Generalversammlung}
\begin{enumerate}
\item Eine ausserordentliche Generalversammlung ist einzuberufen, wenn 20\% aller Mitglieder, die Mehrheit des Vorstandes oder der Fachvereinsrat (FR) dies unter Angabe der zu behandelnden Geschäfte schriftlich beim Vorstand beantragt oder eine vorgehende Generalversammlung dieses beschliesst. Sie ist auch einzuberufen, wenn das Präsidium oder die Quästur vorzeitig aus seinem Amt ausscheidet oder die Ausgaben das Budget absehbar um mehr als 6000 CHF überschreiten und keine kompensierenden Einnahmen vorhanden sind.
\item Eine ausserordentliche Generalversammlung besitzt die selben Rechte wie eine ordentliche Generalversammlung.
\item Ausserordentliche Generalversammlungen müssen mindestens 14 Tage im Voraus durch geeignete Bekanntmachung angekündigt werden.
\item Eine ausserordentliche Generalversammlung sollte nach Möglichkeit während des Semesters abgehalten werden.
\end{enumerate}

\subsection{Einberufung und Abhaltung der Generalversammlung}
\begin{enumerate}
\item Die Bekanntmachung muss allen Mitgliedern per E-Mail oder Brief zugesandt werden. Sie beinhaltet die zu behandelnden Geschäfte und die nötigen Hintergrundinformationen. Dies beinhaltet mindestens die Tätigkeitsberichte des Vorstands, das Protokoll der letzten Generalversammlung, allfällige Statutenänderungen, sowie an der ordentlichen Generalversammlung des Frühjahrssemesters die Rechnung der vergangenen Rechnungsperiode und an der ordentlichen Generalversammlung des Herbstsemesters das Budget der kommenden Rechnungsperiode.
\item Jede ordnungsgemäss einberufene Generalversammlung ist beschlussfähig.
\item Nichtmitglieder dürfen der Generalversammlung als Gäste beiwohnen, sie haben jedoch kein Stimm- und Wahlrecht.
\item Das Präsidium leitet die Generalversammlung. Es kann im Verhinderungsfall einen Stellvertreter bestimmen. Auf Beschluss des Vorstandes, kann eine externe Sitzungsleitung vorschlagen werden. Die neue
Sitzungsleitung muss von der Generalversammlung zu Beginn der Sitzung bestätigt werden.
\item Der Vorstand ist mit Ausnahme des Präsidiums stimm- und wahlberechtigt.
\item Es wird ein Protokoll geführt, das bis zur nächsten Generalversammlung veröffentlicht wird. Es muss von der nächsten Generalversammlung genehmigt und vom Präsidium und den Protokollierenden unterzeichnet werden. Die genehmigten Protokolle sind für die Mitglieder zugänglich und werden dem VSETH und der GPK zugestellt.
\end{enumerate}


\subsection{Wahlen und Abstimmungen}
\begin{enumerate}
\item Bei Wahlen und Abstimmungen entscheidet grundsätzlich das einfache Mehr, sofern die Statuten keinen anderen Modus vorsehen.
\item Die Abstimmungen und Wahlen erfolgen offen, sofern nicht von mindestens einem Mitglied eine geheime Abstimmung oder Wahl verlangt wird. Sind bei Wahlen mehr Kandidierende als Sitze vorhanden, so hat die Wahl in der Regel schriftlich zu erfolgen.
\item Es ist in keinem Fall möglich, das Stimm- und Wahlrecht auf eine andere Person zu übertragen.
\item Entspricht bei einer Wahl oder Abstimmung die Anzahl der abgegebenen Stimmen nicht jener der Anzahl an Wahl-, resp. Stimmberechtigten, so muss die Wahl oder die Abstimmung nicht wiederholt werden unter der Voraussetzung, dass die Differenz keinen Einfluss auf das Wahl-, resp. Abstimmungsergebnis hat.
\item Bei folgenden Geschäften ist eine Zweidrittelmehrheit erforderlich:
	\begin{itemize}
	\item Statutenänderungen
  \item Wahl eines Ehrenmitglieds
	\item Ausschluss von Ehren- und Passivmitgliedern
	\end{itemize}
\item Es gelten die Wahl- und Abstimmungsverfahren gemäss Reglement über die Verfahren der Mitwirkung und das Öffentlichkeitsprinzip im VSETH.
\end{enumerate}


\subsection{Berechnung von Mehrheiten}
\begin{enumerate}
\item Wird nur das einfache Mehr verlangt, so entscheidet die einfache Mehrheit der gültigen Stimmen. Bei Stimmengleichheit hat das Präsidium der VCS den Stichentscheid.
\item Das absolute Mehr berechnet sich aus der nächsthöheren ganzen Zahl der durch zwei geteilten Anzahl stimmberechtigter Anwesender.
\item Das Zweidrittelmehr ist die aufgerundete ganze Zahl von 2/3 der stimmberechtigten Anwesenden.
\item Sehen die Statuten ein einfaches Mehr vor, werden Enthaltungen und ungültige Stimmen nicht gezählt. Ist ein absolutes oder Zweidrittelmehr vorgesehen, so gelten in Abstimmungen Enthaltungen sowie ungültige Stimmen als Nein-Stimmen, in Wahlen werden Enthaltungen und ungültige Stimmen bei der Berechnung des Mehres jedoch nicht gezählt.
\end{enumerate}

\subsection{Geschäfte}
\begin{enumerate}
\item Die ordentliche Generalversammlung entscheidet über die Entlastung des Präsidiums und des Vorstandes und ausserdem über alle Belange, welche nicht explizit anderen Organen übertragen sind. Die ordentliche Generalversammlung des Frühjahrssemesters entscheidet über die Genehmigung der Rechnung der vergangenen Rechnungsperiode. Die ordentliche Generalversammlung des Herbstsemesters entscheidet über die Genehmigung des Budgets der nächsten Rechnungsperiode. Änderungen des Budgets können jedoch von jeder Generalversammlung beschlossen werden.
\item Anträge von mehr als 500 CHF und Statutenänderungen müssen mindestens 14 Tage vor der ordentlichen Generalversammlung dem Vorstand schriftlich vorgelegt und begründet werden. Sonstige Anträge können mündlich an der ordentlichen Generalversammlung vorgetragen werden. Über jeden Antrag wird abgestimmt.
\item Die Entscheidungen der Generalversammlung sind für alle Vereinsmitglieder, insbesondere den Vorstand, bindend.
\item Die ordentliche Generalversammlung wählt den Vorstand für ein Semester. Die ordentliche Generalversammlung des Frühjahrssemesters wählt die Rechnungsrevisoren für eine Abrechnungsperiode. Wiederwahl ist möglich.
\item Das Präsidium und die Quästur werden einzeln von der Generalversammlung gewählt. Der restliche Vorstand und das Vizepräsidium werden \textit{in corpore} gewählt, es sei denn, die Generalversammlung beschliesst einen anderen Wahlmodus.
\item Nachwahlen an ausserordentlichen Generalversammlungen sind möglich.
\item Der Vorstand hat das Recht, zwischen zwei ordentlichen Generalversammlungen provisorisch Vorstandsmitglieder aufzunehmen. Diese haben jedoch kein Stimmrecht im Vorstand und müssen sich spätestens an der nächsten ordentlichen Generalversammlung zur Wahl stellen.

\end{enumerate}


\section{Vorstand}

\subsection{Mitglieder}
\begin{enumerate}
\item Der Vorstand besteht aus mindestens drei Vereinsmitgliedern. Es müssen die Ämter Präsidium, Quästur und Hochschulpolitik von mindestens je einem Vorstandsmitglied besetzt sein.
\item Folgende Vorstandsposten müssen von verschiedenen Personen besetzt sein:
	\begin{itemize}
	\item Präsidium
	\item Vizepräsidium
	\item Quästur
	\end{itemize}
\item Im Weiteren sind folgende Personen stimmberechtigte Mitglieder des Vorstandes, falls die Posten besetzt sind:
	\begin{itemize}
	\item die Präsidien der einzelnen Kommissionen mit Ausnahme der Chemtogether-Kommission 
	\item IT-Verantwortliche/r
	\item SchriftführerIn
	\item Beauftragte/r für Studentisches
	\end{itemize}
\item Der Vorstand besteht aus maximal 13 stimmberechtigten Personen.
\item Vorstandsmitglieder können in Abwesenheit gewählt werden, wenn eine schriftliche Zustimmung zur Wahl vorliegt.
\item Der Vorstand wird an jeder ordentlichen Generalversammlung bis zur nächsten ordentlichen Generalversammlung gewählt.
\item Der Vorstand kann für seine Arbeit freie Mitarbeitende ohne Stimmrecht hinzuziehen.
\end{enumerate}

\subsection{Aufgaben}
\begin{enumerate}
\item Der Vorstand ist im Sinne des Vereinszwecks tätig.
\item Er leitet als Exekutive den Verein, führt die Geschäfte und vollzieht die Beschlüsse der Generalversammlung.
\item Er stellt den Mitgliedern der VCS ein Informationsorgan zur Verfügung.
\item Er sorgt für eine satzungsgemässe Einberufung, Bekanntmachung und Durchführung der Generalversammlung sowie die Durchführung von Urabstimmungen.
\item Jedes Vorstandsmitglied ist verpflichtet, an den Generalversammlungen teilzunehmen und über seine Aktivitäten zu berichten. Im Verhinderungsfall kann ein anderes Vorstandsmitglied diesen Bericht stellvertretend vortragen.
\end{enumerate}

\subsection{Vorstandssitzungen}
\begin{enumerate}
\item Der Vorstand trifft sich während des Semesters so oft es die Geschäfte erfordern, jedoch mindestens einmal pro Monat.
\item Die Sitzungsleitung obliegt dem Präsidium.
 \item  Jedes Vorstandsmitglied ist berechtigt eine Vorstandssitzung einzuberufen. 
\item Der Vorstand ist beschlussfähig, wenn mindestens die Hälfte seiner stimmberechtigten Mitglieder anwesend sind. Zur Beschlussfassung sind die Stimmen der Mehrheit der anwesenden Vorstandsmitglieder nötig. Jedes Vorstandsmitglied hat eine Stimme. Bei Stimmengleichheit hat das Präsidium den Stichentscheid.
\item Jedes Vorstandsmitglied ist verpflichtet, an den Vorstandssitzungen teilzunehmen und über seine Aktivitäten zu berichten.
\item Die Sitzungen des Vorstands sind für Mitglieder zugänglich. Die Termine werden auf der Homepage der VCS bekannt gegeben. Die Sitzungsleitung darf Gäste einladen.
\item Auf Antrag von einem oder mehreren Vorstandsmitglieder können die stimmenlosen Beisitzer von einzelnen vertraulichen Traktanden der Vorstandsitzungen ausgeschlossen werden.
\item Es wird ein Protokoll geführt, das vom Vorstand genehmigt werden muss. Die genehmigten Protokolle sind für Mitglieder zugänglich und werden dem VSETH und der GPK zugestellt.
\end{enumerate}

\subsection{Schriftliche Beschlussverfahren}
\begin{enumerate}
\item Ist die mündliche Beratung einer Angelegenheit nicht erforderlich oder muss diese Angelegenheit vor Abhaltung der nächsten Vorstandssitzung erfolgen, so kann das Präsidium oder ein anderes Vorstandsmitglied die Zustimmung des Vorstandes auf schriftlichem Weg einholen.
\item Bei einem schriftlichem Beschlussverfahren entscheidet das relative Mehr der abgegebenen Stimmen.
\item Ein schriftliches Beschlussverfahren behält lediglich dann seine Gültigkeit, wenn mindestens $75\%$ der Vorstandsmitglieder abgestimmt haben.
\item Schriftliche Beschlussverfahren dauern mindestens 24 Stunden. Erfordert die Angelegenheit eine schnellere Behandlung, so kann das Präsidium diese Mindestdauer aufheben.
\item Werden im Rahmen des Beschlussverfahrens Ausgaben von mehr als 100.- CHF gesprochen, so muss für einen gültigen Beschluss eine Antwort der Quästur erfolgt sein.
\item Widerspricht mindestens ein Vorstandsmitglied der Durchführung als schriftliches Beschlussverfahren, so muss die Angelegenheit mündlich behandelt werden.
\item Schriftliche Beschlüsse sind bei der folgenden Vorstandssitzung im Protokoll aufzulisten.
\end{enumerate}

\subsection{Konstituierung}
\begin{enumerate}
\item Der Vorstand konstituiert sich selbst und regelt seine Aufgabenteilung intern.
\item Die Amtsübergabe findet an der ersten Vorstandssitzung nach einer Generalversammlung statt. Die Quästur übergibt sein Amt erst am Ende des laufenden Quartals.
\item In Abwesenheit des Präsidiums übernimmt das Vizepräsidium alle Rechte und Pflichten des Präsidiums.
\end{enumerate}

\subsection{Finanzen und Zeichnungsberechtigungen}
\begin{enumerate}
  \item Die Quästur erstellt den Budgetvorschlag sowie den Jahresabschluss und bespricht den Budgetvorschlag sowie den Jahresabschluss vor der jeweils zuständigen Generalversammlung mit dem Vorstand.
  \item Das auf der Generalversammlung beschlossene Budget ist grundsätzlich bindend. Der Vorstand verfügt über die von der Generalversammlung beschlossenen Budgetposten im Sinne des Vereinszwecks gemäss den Bestimmungen in Ziffer 3.
  \item 
    \begin{enumerate}
      \item Für Projekte, deren Verträge Ausgaben (Aufwand) von bis zu 700.- CHF innerhalb eines Budgetpostens auslösen, entscheidet und verfügt grundsätzlich jedes Vorstandsmitglied innerhalb seines Ressorts. Sind solche Ausgaben ressortübergreifend, entscheiden und verfügen die betroffenen Vorstandsmitglieder gemeinsam.

        Über Verträge, die höhere Ausgaben pro Projekt innerhalb eines Budgetpostens auslösen, sowie solche, die ein Vorstandsmitglied dem Vorstand zur Entscheidung unterbreitet, entscheidet der Gesamtvorstand und verfügt daraufhin das zuständige Vorstandsmitglied.
      \item Über Verträge, die Ausgaben auslösen, die den dafür vorgesehenen Budgetposten überschreiten, entscheidet der Gesamtvorstand bis zu einer Überschreitung von 50 \% des Budgetpostens oder -falls mehr- bis zu 1000.- CHF und verfügt daraufhin das zuständige Vorstandsmitglied. Über darüber hinausgehende Budgetüberschreitungen entscheidet eine Generalversammlung.
      \item Über Verträge, die Ausgaben auslösen, die nicht von einem Budgetposten erfasst werden, entscheidet bis zu einem Einzelbetrag von maximal 1000.- CHF pro Projekt der Gesamtvorstand und verfügt daraufhin das zuständige Vorstandsmitglied. Unter Anwesenheit der Quästur und bei zu erwartenden Kompensationseinnahmen kann die Obergrenze auf einen Betrag von 5000.- CHF pro Projekt angehoben werden.
    \end{enumerate}
  \item Über alle vorhersehbaren Ausgaben oberhalb von 100.- CHF pro Projekt ist die Quästur vorab zu informieren.
  \item Die Finanzen der VCS werden auf einem oder mehreren separaten Konten geführt.
  \item Für jedes Ereignis mit Einnahmen und Ausgaben ist der Quästur eine ausführliche Abrechnung zu erstellen.
  \item Sowohl die Quästur als auch das Präsidium der VCS besitzen für alle Konten die Berechtigung zur Einzelunterschrift. Im Falle der Verhinderung der Quästur wird diese durch das Präsidium vertreten.
 \item In allen sonstigen Fällen wird der Verein durch das Präsidium allein oder durch die Quästur zusammen mit einem weiteren Vorstandsmitglied nach Aussen hin juristisch vertreten.

 \item Töpfe im Budget der VCS
    \begin{enumerate}
        \item Der VCS Vorstand kann Ausgaben, welche nicht unter einen Budgetposten fallen, aus Töpfen tätigen. Diese müssen mit dem Zweck der VCS vereinbar sein.
        \item Die Definition der Töpfe und deren Gesamthöhe werden im jährlichen Budget festgelegt.
        \item Ein Antrag auf Unterstützung besteht aus einer detaillierten Beschreibung sowie einer Kostenaufschlüsslung, welche dem Vorstand vorgelegt werden muss.
        \item Es werden pro Antrag maximal CHF 500 vergeben.
        \item Über den Antrag wird im Vorstand abgestimmt, sobald eine detaillierte Beschreibung sowie die Kostenaufschlüsslung vorliegen.
        \item Eine Liste der genehmigten Anträge mit den beantragten Summen ist der GV bei der Genehmigung der Rechnung vorzulegen.
        \item Events, welche aus Töpfen (mit-)finanziert werden, müssen grundsätzlich für jedes VCS-Mitglied zugänglich sein.
    \end{enumerate}
\end{enumerate}

\subsection{Entschädigung}
\begin{enumerate}
\item Die Vorstandstätigkeit ist ehrenamtlich. Es gibt keine finanzielle Entschädigung. Abschiedsgeschenke an Vorstandsmitglieder sind eine Ausnahme und im Rahmen des Budgets zulässig.
\item Der Vorstand kann Vorstandsmitgliedern Spesen, die im Zusammenhang mit der Ausübung des Amtes stehen, erstatten. Diese Spesen sind zu belegen.
\end{enumerate}

\section{Kommissionen}

\subsection{Grundlage}
\begin{enumerate}
\item Die Generalversammlung kann Kommissionen bestellen und wieder auflösen. Sie legt für jede einzelne deren Rechte und Pflichten fest.
\item Der Verein haftet für alle Verbindlichkeiten seiner Kommissionen.
\end{enumerate}

\subsection{Kommissionsreglement}
\begin{enumerate}
\item Das Kommissionsreglement regelt Organisation und Tätigkeit der Kommission.
\item Das Kommissionsreglement wird durch die Generalversammlung verabschiedet.
\item Die Statuten sind dem Kommissionsreglement übergeordnet. Das Reglement darf nicht in Widerspruch zu diesen Statuten stehen.
\end{enumerate}

\subsection{Mitglieder}
\begin{enumerate}
\item Das Kommissionspräsidium entscheidet über die Kommissionsmitglieder. Die Kommissionsarbeit ist grundsätzlich offen für alle Mitglieder. Ausnahmen und Ausschluss obliegen dem Kommissionspräsidium.
\item Die Kommissionspräsidien sind durch die Generalversammlung zu bestimmen. Ihre Amtszeit erstreckt sich, mit Ausnahme der Chemtogether-Kommission, bis zur nächsten ordentlichen Generalversammlung. Für die Chemtogether-Kommission wird zusätzlich zum Präsidium auch die Quästur an der Generalversammlung gewählt.
\item Zur Rekrutierung neuer Kommissionsmitglieder müssen zumindest anschliessend zur Generalversammlung Listen aufliegen.
\end{enumerate}

\subsection{Organisation}
\begin{enumerate}
\item Die Kommission lädt den VCS-Vorstand zu allen Sitzungen ein, erstattet ihm Bericht und stellt ihm ihre Protokolle zu.
\item Die Kommissionspräsidien legen an der Generalversammlung einen Tätigkeitsbericht vor.
\end{enumerate}


\subsection{Finanzen}
\begin{enumerate}
\item Die Chemtogether-Kommission hat eine eigene Rechnungsführung. Die Rechnungsführungen der anderen Kommissionen obliegen der Quästur der VCS.
\item Die Beiträge der VCS werden im Budget festgelegt.
\item Die Rechnungen aller Kommissionen sind Bestandteil der Rechnung der VCS und werden durch die Revisoren geprüft.
\end{enumerate}

\section{Rechnungsrevisoren}

\subsection{Zusammensetzung}
\begin{enumerate}
\item Die Revisorengruppe besteht aus mindestens zwei Personen. Vorstandsmitglieder sowie Präsidium und Quästur der Chemtogetherkommission können der Revisorengruppe nicht angehören.
\end{enumerate}

\subsection{Aufgabe}
\begin{enumerate}
\item Die Rechnungsrevisoren prüfen die Jahresrechnung des Vereins und seiner Kommissionen unabhängig und neutral.
\item Sie erstatten der Generalversammlung schriftlich Bericht und stellen bei korrekter Geschäftsführung Antrag auf Entlastung des Vorstandes.
\end{enumerate}


\section{Vertretungen in anderen Organisationen}
\label{sec:Vertretungen}
\subsection{Vertretungen}
\begin{enumerate}
\item Der Verein kann in andere Organisationen Vertreter abordnen, die dort seine Interessen wahren.
\end{enumerate}

\subsection{Vertretungen im D-CHAB}
\begin{enumerate}
\item Die Generalversammlung wählt die studentischen Vetretungen folgender Gremien:
	\begin{itemize}
	\item Departementskonferenz D-CHAB (DK)
	\item Unterrichtskommission für die Studiengänge Chemie, Chemie- und Bioingenieurwissenschaften, Biochemie - Chemische Biologie (UK-C)
	\item Unterrichtskommission für den Studiengang Interdisziplinäre Naturwissenschaften (UK-N)
	%\item Berufungskommission
	\end{itemize}
\item
	\begin{enumerate}
	\item Die Generalversammlung wählt sowohl für die DK, wie auch für die UK-C und die UK-N sechs Vertretungen. Sämtliche restliche Kandidierenden werden als Stellvertreter geführt.
	\item Weitere Stellvertreter können vom Präsidium der Hochschulpolitischen Kommission der VCS in Abstimmung mit dem Vorstand bestimmt werden.
	\item Sowohl in der DK wie auch in der UK-C und der UK-N müssen jeweils mindestens zwei Vertretungen von Vorstandsmitgliedern besetzt sein.
	\item Sofern die Kandidierendenlisten es erlauben, muss jeder an der jeweiligen UK vertretenden Studiengang durch eine/n Studierende/n dieses Studiengangs repräsentiert werden.
	\item Vertretungen in Berufungskommissionen werden vom Vorstand gewählt.
	\end{enumerate}
\item Der Vertreter der Studierenden des D-CHAB in der Bibliothekskommission gemäss Artikel~5 des Reglements für das ETH Informationszentrum Chemie Biologie Pharmazie ist \textit{ex officio} das Präsidium der VCS gemeinsam mit einem Vertreter des APVs.
\item Der Vertreter der Studierenden in der Notenkonferenz gemäss Artikel 14.2 der Geschäftsordnung des D-CHAB ist \textit{ex officio} das Präsidium der VCS. Vertretung ist im Verhinderungsfall möglich.
\end{enumerate}

\subsection{Vertretungen im VSETH}
\begin{enumerate}
\item Die VCS entsendet Vertreter in den Mitgliederrat (MR) und in den Fachvereinsrat (FR) des VSETH gemäss Art. 17. Abs. 1 der VSETH-Statuten und Art. 1 des Gremienreglement über den Fachvereinrat.
\item Ein Vorstand für Hochschulpolitik sowie entweder das Präsidium der VCS oder ein anderes ordentliches Vorstandsmitglied, welches an der GV für die Vertretung im FR gewählt wurde, teilen die Ämter des Delegierten und Stellvertreters im FR untereinander auf.
\item Die restlichen der VCS zugeteilten Sitze des MR werden von der ordentlichen Generalversammlung bis zur nächsten ordentlichen Generalversammlung gewählt. Wiederwahl ist möglich. Kandidierende, welche nicht zu Delegierten des MR gewählt wurden, werden als Ersatzdelegierte geführt.
\item Weitere Ersatzdelegierte können vom Präsidium der Hochschulpolitischen Kommission der VCS in Abstimmung mit dem Vorstand bestimmt werden.
\end{enumerate}

\subsection{Berichterstattung}
\begin{enumerate}
\item Die Vertretungen sind verpflichtet dem Vorstand Bericht zu
erstatten.
\item Mindestens ein Mitglied jeder Vertretung nimmt an den Generalversammlungen teil. Vertretung ist im Verhinderungsfall möglich.
\end{enumerate}

\section{Schlussbestimmungen}

\subsection{Statuten} \label{sec:statutenAenderung}

\begin{enumerate}
\item Zur Revision der Statuten sind zwei Drittel aller Stimmen der an einer Generalversammlung anwesenden Stimmberechtigten notwendig.
\item Eine Ausnahme bilden Artikel \ref{sec:statutenAenderung} und Artikel \ref{sec:vereinsAufloesung} Absatz 1, für die eine Zweidrittelmehrheit aller Mitglieder notwendig ist.
\end{enumerate}

\addtocounter{article}{-2} %quick fix so it does not skip two numbers (prompted apparently by the refs in Art. 36, no idea why

\subsection{Vereinsauflösung}
\label{sec:vereinsAufloesung}
\begin{enumerate}
\item Für die Auflösung der VCS sind die Stimmen von 2/3 aller Mitglieder notwendig. Der Antrag auf eine Urabstimmung über die Auflösung der VCS muss von mindestens 10\% aller Mitglieder unterzeichnet sein.
\item Bei Auflösung der VCS wird das Vermögen dem VSETH zu treuen Händen übergeben, bis sich eine Vereinigung oder Organisation mit ähnlichen Zielsetzungen bildet.
\end{enumerate}

\subsection{Inkraftsetzung}
\begin{enumerate}
\item Die vorliegenden Statuten wurden von der Generalversammlung an ihrer Sitzung am \thedate \ einer Revision unterzogen und genehmigt. Sie ersetzen die Statuten vom \theolddate \ und treten ab dem \thedateplus \ in Kraft.
\end{enumerate}